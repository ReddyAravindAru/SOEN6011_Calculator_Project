%%%%%%%%%%%%%%%%%%%%%%%%%%%%%%%%%%%%%%%%%%%
%%% DOCUMENT PREAMBLE %%%
\documentclass[12pt]{report}
\usepackage[english]{babel}
%\usepackage{natbib}
\usepackage{url}
\usepackage{algorithm}
\usepackage[noend]{algpseudocode}
\usepackage[utf8x]{inputenc}
\usepackage{amsmath}
\usepackage{graphicx}
\graphicspath{{images/}}
\usepackage{parskip}
\usepackage{fancyhdr}
\usepackage{vmargin}
\setmarginsrb{3 cm}{2.5 cm}{3 cm}{2.5 cm}{1 cm}{1.5 cm}{1 cm}{1.5 cm}

\title{Software Engineering Process - SOEN 6011}	
% Title
\author{ }	
% Author
\date{}
% Date

\makeatletter
\let\thetitle\@title
\let\theauthor\@author
\let\thedate\@date
\makeatother

\pagestyle{fancy}
\fancyhf{}
\rhead{\theauthor}
\lhead{\thetitle}
\cfoot{\thepage}
%%%%%%%%%%%%%%%%%%%%%%%%%%%%%%%%%%%%%%%%%%%%
\begin{document}

%%%%%%%%%%%%%%%%%%%%%%%%%%%%%%%%%%%%%%%%%%%%%%%%%%%%%%%%%%%%%%%%%%%%%%%%%%%%%%%%%%%%%%%%%

\begin{titlepage}
\centering
    \vspace*{0.5 cm}
   % \includegraphics[scale = 0.075]{bsulogo.png}\\[1.0 cm]	% University Logo
\begin{center}    \textsc{\Large   }\\[2.0 cm]	\end{center}% University Name
\textsc{\Large Individual Report  }\\[0.5 cm]	% Course Code
\rule{\linewidth}{0.2 mm} \\[0.4 cm]
{ \huge \bfseries \thetitle}\\
\rule{\linewidth}{0.2 mm} \\[1.5 cm]

\begin{minipage}{0.4\textwidth}
\begin{center} \large
%	\emph{Submitted To:}\\
%	Name\\
          % Affiliation\\
           %contact info\\
\end{center}
\end{minipage}~
\begin{center}{}
            
\begin{center} \large
\emph{Submitted By :} \\
Aravind Ashoka Reddy  \\
40103248 \\

\end{center}
           
\end{center}\\[2 cm]


    
    
    
    

\end{titlepage}

%%%%%%%%%%%%%%%%%%%%%%%%%%%%%%%%%%%%%%%%%%%%%%%%%%%%%%%%%%%%%%%%%%%%%%%%%%%%%%%%%%%%%%%%%

\tableofcontents
\pagebreak

%%%%%%%%%%%%%%%%%%%%%%%%%%%%%%%%%%%%%%%%%%%%%%%%%%%%%%%%%%%%%%%%%%%%%%%%%%%%%%%%%%%%%%%%%
\renewcommand{\thesection}{\arabic{section}}
\section{Problem 1}
\textbf{Function assigned: Γ(x)}\\
The (complete) gamma function     is defined to be an extension of the factorial to complex and real number arguments. It is related to the factorial by\\



The gamma function is given by this integral for all positive xx. Then there exists an analytic function with domain C∖{0,−1,−2,…},  such that its restriction to positive axis coincides with the value of that integral. \\
\begin{center}
	C/{ n∈Z, n≤0}\\
\includegraphics[]{Untitled1.png}\\
\end{center}
Properties of the Gamma Function\\
   \includegraphics[]{Untitled.png}
 

   
















\newpage
\section{Problem 2}
\begin{center}
    

\begin{center}
    \textbf{Requirements}\large
\end{center}
\textbf{R-1} : When the input for Gamma function is received, the user defined function checks the input if it is 1, then outputs the gamma value as 1.\\
\textbf{R-2} : When the input ½ is received, the user defined function checks if the input is equal to ½, if yes outputs the gamma value as √\pi.\\


\textbf{R-3} : when the input for the program is a non-positive integer, the function which calculates the gamma value discards the input and prints invalid input.\\



\textbf{R-4} : When the input (x) is a positive integer, the function validates and returns a gamma value approximate to (x-1)!\\




\newpage
\section{Problem 3}
\textbf{Algorithm with Stirling’s Approximation}\\
This algorithm using sterling's approximation for giving precision. Technically sterling's approximation gives the best precision for any gamma value.

\textbf  {\large Algorithm A.{\rom 1} }: Calculating Gamma value using sterling's precision\\

\begin{algorithm}
\caption{My algorithm1}\label{euclid}
\begin{algorithmic}[1]
\Procedure{CalculateGamma}{x}
\State $\textit{stringlen} \gets \text{length of }\textit{string}$
\State $E \gets \textit{2.7182818284}$
\State $PI \gets \textit{3.1415926535}$




\BState \emph{loop}:
\If {$\textit{string}(i) = \textit{path}(j)$}
\State $j \gets j-1$.
\State $i \gets i-1$.
\State \textbf{goto} \emph{loop}.
\State \textbf{close};
\EndIf
\State $i \gets i+\max(\textit{delta}_1(\textit{string}(i)),\textit{delta}_2(j))$.
\State \textbf{goto} \emph{top}.

\EndProcedure

\Procedure{squarerootCalc}{x}

\If{ Val == 0} \Return 0 
\EndIf
\State $last \gets \textit{0.0}$
\State $res \gets \textit{1.0}$

\BState \emph{loop}:
\If {$\textit{res} == \textit{last}$}
     

\State $last \gets res$.
\State $res \gets (res {+} 2 \div res) \div 2  $.
\State \textbf{goto} \emph{loop}.
\State \textbf{close};
\EndIf
\State \Return res


\EndProcedure

\Procedure{powerCalc}{N,x}
\If{ x == 0} \Return 1 


\EndIf





\EndProcedure
\end{algorithmic}
\end{algorithm}

\newpage
 
\begin{thebibliography}{111}
   
  \bibitem{ACMT}
http://mathworld.wolfram.com/GammaFunction.html

%if the "underfill \hbox" warning bothers you uncomment the following line
%\raggedright
\bibitem{ACAMP}
    https://www.probabilitycourse.com/chapter4/4_2_4_Gamma_distribution.php
  
\bibitem{Gro01} 
   https://math.stackexchange.com/questions/705103/what-is-the-domain-of-gamma-function

\end{thebibliography}
\end{document}

%This template was created by Roza Aceska.